\section{原稿用紙}
\subsection{タイトルその他(1ページ目上部)に関して}
\label{subsec:intro}
技術研究報告の1ページ目上部には、タイトル、発表者氏名、所属、住所、メールアドレス、
キーワードの和文と英文及びあらまし(和文300字程度、英文100語程度)を、それぞれ記述すること。

[特別招待講演]の方は[特別招待講演]、[特別講演]の方は[特別講演]、
[招待講演]の方は[招待講演]、[基調講演]の方は[基調講演]等、
一般の講演以外の方はタイトルの前に[○○講演]と必ず挿入すること。

\section{本文に関して}
本文は\ref{subsec:intro}の「タイトルその他」に続けて記述する。
記述に関しては、このテンプレートファイルを用いて作成するか、
あるいは、任意のA4判の用紙を利用することができる。
その場合には、本文は左右18cm、天地25.5cm以内の長さにおさまるよう行間・字間を調整すること。

\subsection{原稿提出枚数}
連絡用紙に指定の提出枚数が記載してあり、
図・表、写真を含め制限枚数以内で作成すること。
原稿を作成する前に、手持ちの原稿量と制限枚数とを十分勘定して必ず
制限枚数におさまるようご注意すること。
枚数を超過した原稿は受け付けられない。
