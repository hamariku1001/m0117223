%%%%%%%%%%%%%%%%%%%%%%%%%%%%%%%%%%%%%%%%%%%%%%%%%%%%%%%%%%%%%%%%%%%%%%%%%%%%%%
%%% タイトル、著者、所属、概要

% 日本語タイトル
\jtitle{
レトロネーザルアロマを利用した \\
味覚の拡張
}

% 英語タイトル
\etitle{
Expressive Japan \\
Sample Style 
}

% 日本語著者
% 共著者氏名の間隔は ~ や \quad 等で適宜調節のこと。
\jauthor{
濱家陸\({}^\dagger\) \quad
羽田久一\({}^\ddagger\) \quad
}

% 英語著者
\eauthor{
Riku Hamaie\({}^\dagger\) \quad
}

% 日本語所属
\jaffiliation{
\(\dagger\) 東京工科大学 ~~~
	〒192-0982 東京都八王子市片倉町1404-1 \\
}

% 英語所属
\eaffiliation{
\(\dagger\) Graduate School of Bionics, Computer and Media Sciences,
	Tokyo University of Technology \\
\(\ddagger\) School of Media Science,
	Tokyo University of Technology
}

% 連絡先電子メールアドレス
% (
% このサンプルでは「@」を2バイト文字にすることで対応してある。)
\email{
m01172238c@edu.teu.ac.jp
}

% 日本語概要
\jabstract{
風味に着目し,密接に関わっている口内から入る香り(レトロネーザル)を利用した.
レトロネーザルで感じた香りは味を感じ方に大きな影響を与えると考え,嗅覚から感じる風味に焦点をあてた実験を検証した。
口の中からの嗅覚刺激を用いて香りを提示することで, 風味を与えることができるのではないかと考え,香りを閉じこめた直径3センチメートルのゼリーを食べることで,風味を想起させることが出来るかを検討した.
被験者に風味を感じることができるかを判断してもらい,風味提示ゼリーの有用性を示した.	
}

% 英語概要
\eabstract{
This article is to provide LaTeX sample for posting of ``Expressive Japan''.
This sample is created with emphasis on consistency with the sample of
the journal of the Society of Art and Science.
}

% 日本語キーワード
\jkeyword{
嗅覚, 味覚, 香り, 風味
}

% 英語キーワード
\ekeyword{
Smell, taste, fragrance, Flavor 
}
