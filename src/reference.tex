\section{関連研究}

実際にそれらの関係性を
利用した研究は行われている.
角谷らの呼吸と連動した醤油の匂い提示による塩味増強
効果\cite{enmi}ではレトロネーザルに着目しており,刺激提示装置は
呼吸センサを伴う前後鼻腔経路嗅覚デバイスを用いた. 
前鼻腔経路に刺激を提示するときは吸気,
後腔経路に提示するときは呼気に合わせて刺激を
提示することにより,食べ物の風味を増強させることが出
来ることを示した.


また,岡崎らの嗅覚ディスプレイは,レトロネーザルに
に着目し,箸の先から香りをだし口内に入れることで,風味
を増強させることが出来ると示した.\cite{hasi}


一方で鳴海らによるメタクッキー\cite{narumi2}は味覚に対して,オル
ソネーザルと視覚による刺激を用いる. プレーン味のクッ
キーに対して HMD を用いた見た目の違うクッキーに見せ,
オルソネーザルの嗅覚刺激で別の味のクッキーの香りをエ
アポンプによる空気の送風で香りかがせる. 視覚と嗅覚を
用いることでの味の変化がある回答を得ている.


白須らはオルソネーザルからの嗅覚刺激を利用した風味
変容の研究を行っており,「かき氷」を題材とし,味覚変容
の手法を検討している. かき氷のシロップを容器に LED
光源を取り付けることでシロップの色を再現し,スプーン
から香料を出すことで鼻に直接香りを与える. 視覚的に着
色料ではなく光源をを使用することで,リアルタイムで同
じ皿で視覚情報と嗅覚情報を切り替えるシステムを作成し
ている. 視覚と鼻からの嗅覚刺激を用いることでの味覚の
変化の有用性を示した。\cite{fan}\cite{pomp}


横山らは嗅覚を用いて情報の伝達と提示を行うデバイス
システム\cite{hirose}を開発し実験を行った. この実験は両方の鼻に香
りを送り濃度を変化させることで香りの強度,空間情報を
伝達させることを示している.
他にも味覚と嗅覚は複数の感覚と影響し合う研究は行わ
れおり,Nimesha らによる Vocktail\cite{vock}は味覚,嗅覚,視覚を
利用した研究を行っている. 視覚情報として LED を使用
した色の印象,嗅覚として香りを与え,味覚として電気味覚
を使用しているこれらを使い水の風味がどのようにして変
化するのか実験を行った. このシステムは,3 つの感覚を活
用して味覚にちする影響を与えていたことを明らかにして
いる.
