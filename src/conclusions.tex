\section{おわりに}

本研究では,レトロネーザルアロマと嗅覚デバイスを利用して,風味の変化を起こすシステム
の検討を行った.そのためにゼリーを用いた風味変容システムを構築した.
このシステムを用いて,ゼリーに対して香りを付与し,食した時の風味を評価してもらう実験
を行った.その結果,風味の変化を感じているという結果が得られ,口内に入れたときの風味を
認識させることができるという有効性を示した.


この手法は,決まった風味だけでなく様々な風味に対して一瞬の風味を想起させることができ
るのではないかと考えている.具体的には,紅茶の風味やフルーツの風味のような香りが想像す
ることができるものでなら行えるのではないかと考えており,今後の検討にしていきたい.


現段階において,香りを入れる際に,そのたびに注射器で香料を入れ替えなければならない.
また,その提示する香りが混ざらないように香料をいれる注射器の容器を分ける事の重要性を改
めて再確認した.今回の実験では香りが混ざらないように香料が付着したものは香料ごとに分け
保管し,洗浄していた.加えて,一度香りが部屋に充満してしまうと被験者が余計な香りに気づ
き誤差を生じてしまう可能性があった.異なる香りが混ざり合わないよう配慮が必要である.


今後の展望として,今回実験で使用した以外の香料での風味の感じ方を検証していきたいと考
えている.香料にはフルーツだけではなく,
様々な香料での風味の感じ方の比較実験を実践していくことが必要である.
また,味覚や視覚を組み合わせたときに風味の感じ方と味覚への影響の変化があるのではないか
と考えている.そして,換気や部品の保管方法の重要性を再確認したところで,実験が行いやす
くするための実験方法を検討している.


以上で述べたことを実践していくことで,レトロネーザルからの風味を認識させることができ
るのではないかと考えている.また,実際に食べなくても嗅覚だけで味を想起することが出来る
体験を感じさせることができる.
