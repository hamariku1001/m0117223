%%%%%%%%%%%%%%%%%%%%%%%%%%%%%%%%%%%%%%%%%%%%%%%%%%%%%%%%%%%%%%%%%%%%%%%%%%%%%%
%%% タイトル、著者、所属、概要

% 日本語タイトル
\jtitle{
レトロネーザルアロマを利用した \\
味覚の拡張(\LaTeX 版)
}

% 英語タイトル
\etitle{
Expressive Japan \\
Sample Style (\LaTeX Version)
}

% 日本語著者
% 共著者氏名の間隔は ~ や \quad 等で適宜調節のこと。
\jauthor{
濱家陸\({}^\dagger\) \quad
\({}^\ddagger\) \quad
\({}^\ddagger\)
}

% 英語著者
\eauthor{
Riku Hamaie\({}^\dagger\) \quad
Jiro MEDIA\({}^\ddagger\) \quad
Saburo GAME\({}^\ddagger\)
}

% 日本語所属
\jaffiliation{
\(\dagger\) 東京工科大学 ~~~
	〒192-0982 東京都八王子市片倉町1404-1 \\
\(\ddagger\) 東京工科大学メディア学部 ~~~
	〒192-0982 東京都八王子市片倉町1404-1
}

% 英語所属
\eaffiliation{
\(\dagger\) Graduate School of Bionics, Computer and Media Sciences,
	Tokyo University of Technology \\
\(\ddagger\) School of Media Science,
	Tokyo University of Technology
}

% 連絡先電子メールアドレス
% (
% このサンプルでは「@」を2バイト文字にすることで対応してある。)
\email{
taro@gamescience.jp
}

% 日本語概要
\jabstract{
嗅覚は味の知覚に密接に関わり影響を与えている.
風味に着目し,密接に関わっている口内から入る香り(レトロネーザル)を利用した.
レトロネーザルで感じた風味は味を感じ方に大きな影響を与えると考え,舌で感じる味覚ではなく嗅覚から感じる風味に焦点をあてた実験を検証した。
口の中からの嗅覚刺激を用いて香りを提示することで, 風味を与えることができるのではないかと考え,香りを閉じこめた直径 3 センチメートルのゼリーを食べることで,風味を想起させることが出来るかを検討した.
被験者に風味を感じることができるかを判断してもらい,風味提示ゼリーの有用性を示した.	
}

% 英語概要
\eabstract{
This article is to provide LaTeX sample for posting of ``Expressive Japan''.
This sample is created with emphasis on consistency with the sample of
the journal of the Society of Art and Science.
}

% 日本語キーワード
\jkeyword{
LaTeX, 論文, テンプレート, 学会
}

% 英語キーワード
\ekeyword{
LaTeX, article, template, society
}
