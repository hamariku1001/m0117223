\subsection{図表}

図や表を論文本体に掲載する場合には、すべての図表を本文から引用し、
適切な位置(引用された文章に近い位置)に表示すること。
すべての図表には通し番号および題名をつけること。

LaTeX で論文を執筆する場合には、
図を EPS ファイルや PDF ファイル等の画像ファイルとして用意し、
figure 環境中にて includegraphics を用いて本文中に挿入する。
図には必ず label を付加し、本文から label を用いて参照する。
本サンプルの場合には、「図\ref{fig:sample}参照」というように記載すれば、
適切に label から図番号を設定してくれるはずである。
以下にその一例をしめす。

\begin{figure}[H]
 \centering
 \includegraphics[width=40mm]{fig-sample.eps}
 \caption{\small{図の挿入例。}}
 \label{fig:sample}
\end{figure}

表についても同様に、label を付加し、本文から label を用いて参照する。
一例として、以下の表 \ref{tab:sample} をご参照いただきたい。

\begin{table}[H]
 \caption{\small{表の挿入例。}}
 \centering
 \begin{tabular}{|c|c|c|c|}
	\hline
		& 数学	& 英語	& 国語	\\ \hline
	太郎	& 68	& 91	& 34	\\
	次郎	& 53	& 12	& 97	\\ \hline
 \end{tabular}
 \label{tab:sample}
\end{table}

本サンプルでは、図表の出現が tex ファイルの記述箇所と同一となるように、
figure 環境や table 環境のオプションに「H」を用いているが、
このオプションは適宜変更しても構わない。

また、図表を一段組で大きく描画したい場合は、
figure 環境や table 環境の末尾にアスタリスクをつけた
「\verb+\begin{figure*} 〜 \end{figure*}+」や
「\verb+\begin{table*} 〜 \end{table*}+」を用いる。
図 \ref{fig:sample-big} でその例を示す。

\begin{figure*}[ht]
 \centering
 \includegraphics[width=80mm]{fig-sample.eps}
 \caption{\small{一段組での図の挿入例。}}
 \label{fig:sample-big}
\end{figure*}

\section{数式}
数式のインラインモードは \(x^2 + y^2 \leq 1\) のように表示させることができる。
インラインモードで「\verb+$...$+」を使うやり方は、
近年の LaTeX ではあまり推奨されていないが、その利用は妨げない。

ディスプレイ数式モードを利用する際に推奨するのは equation 環境である。
\begin{equation}
	\bA_p = \frac{\bA\cdot\bB}{|\bB|^2}\bB .
	\label{eq:samp1}
\end{equation}
数式の参照は「\verb+\ref+」ではなく「\verb+\eqref+」を用いる。
上記の数式を参照すると「式\eqref{eq:samp1}」となる。
このように、\verb+\eqref+ を用いた場合は数式中と同じ様式の括弧がつく。

また、複数行にわたる数式を表示したい場合は align 環境を用いることを推奨する。
以下の式\eqref{eq:samp2}にその例を示す。

\begin{align}
	& \begin{bmatrix}
	a_{11} & a_{12} & \cdots & a_{1n} \\
	a_{21} & a_{22} & \cdots & a_{2n} \\
	\vdots & \vdots & \ddots & \vdots \\
	a_{m1} & a_{m2} & \cdots & a_{mn} \\
	\end{bmatrix}
	\otimes
	\begin{bmatrix}
	b_{11} & b_{12} & \cdots & b_{1n} \\
	b_{21} & b_{22} & \cdots & b_{2n} \\
	\vdots & \vdots & \ddots & \vdots \\
	b_{m1} & b_{m2} & \cdots & b_{mn} \\
	\end{bmatrix} \notag \\
	& \qquad \qquad = \sum_{i}^{m}\sum_{j}^{n}a_{ij}b_{ij} .
	\label{eq:samp2}
\end{align}

eqnarray 環境は、最近の LaTeX では幾つかのパッケージと同時に利用すると
問題が発生することがあるため、利用は推奨しない。

\subsection{参考文献}
参考文献は本文の後に全部まとめて列挙する。
すべての参考文献は本文中で引用する。すべての参考文献には通し番号をつける。

本稿の末尾に、英語論文と日本語論文の参考文献の一例 \cite{Ito04} を示す。
原則として、著者名、タイトル、掲載誌、(論文の場合には巻と号)、
ページ数、発行年を記載すること。
著書の場合には、著書を特定する情報(出版社、ISBNなど)もできる限り記載すること。
なおウェブサイト等\cite{ArtScience}を引用する場合には、この限りではない。

本ファイルは\BibTeX を利用することを想定したサンプルとなっているが、
\BibTeX を利用せずに参考文献リストを記述する場合は、
「BibTeXを利用しない場合」と記されている箇所のコメントアウトされている部分を
参照のこと。

\section{PDF化について}
\begin{itemize}
\item セキュリティ設定は無効にすること。
\item ページ番号は挿入しないこと。
\end{itemize}

\section{まとめ}

本稿では、映像情報・芸術科学フォーラム投稿用の \LaTeX 版サンプルを提供した。
