\section{関連研究}

実際にそれらの関係性を
利用した研究は行われている.
鳴海らの呼吸と連動した醤油の匂い提示による塩味増強
効果ではレトロネーザルに着目しており, 刺激提示装置は
呼吸センサを伴う前後鼻腔経路嗅覚デバイスを用いた. 嗅
覚刺激として市販の醤油を用いた. 提示の際には, 綿に染
み込ませたい耐熱性のプラスチック製のボトルに入れてい
る. 効率的に香気成分を提示するために, 実験中, 匂い瓶及
び水の入った瓶は 65 ℃に保温している. 味覚刺激用の食
塩水は, 市販の食塩と純粋を用いて作成している. 嗅覚提
示の行い方は, PC ディスプレイが設置された机の前の椅
子に着席し, 嗅覚ディスプレイに接続されたヘッドセット
を装着した. この際に, 鼻孔に前鼻孔刺激と空気を配給す
るシリコンチューブを 5mm 程度挿入する. また, こう鼻孔
刺激と空気を配給するために, 実験参加者自身にポンプに
連結させたストローを口腔に挿入させ, できるだけ咽頭に
匂いが届くようにした. 前鼻腔経路に刺激を提示するとき
は吸気, 後腔経路に提示するときは呼気に合わせて刺激を
提示することにより, 食べ物の風味を増強させることが出
来ることを示した.[1]


また、岡崎らの嗅覚ディスプレイは, レトロネーザルに
に着目し, 箸の先から香りをだし口内に入れることで, 風味
を増強させることが出来ると示した.[2]


一方で鳴海らによるメタクッキーは味覚に対して, オル
ソネーザルと視覚による刺激を用いる. プレーン味のクッ
キーに対して HMD を用いた見た目の違うクッキーに見せ,
オルソネーザルの嗅覚刺激で別の味のクッキーの香りをエ
アポンプによる空気の送風で香りかがせる. 視覚と嗅覚を
用いることでの味の変化がある回答を得ている.[3]


白須らはオルソネーザルからの嗅覚刺激を利用した風味
変容の研究を行っており, 「かき氷」を題材とし, 味覚変容
の手法を検討している. かき氷のシロップを容器に LED
光源を取り付けることでシロップの色を再現し, スプーン
から香料を出すことで鼻に直接香りを与える. 視覚的に着
色料ではなく光源をを使用することで, リアルタイムで同
じ皿で視覚情報と嗅覚情報を切り替えるシステムを作成し
ている. 視覚と鼻からの嗅覚刺激を用いることでの味覚の
変化の有用性を示した。[4][5]


横山らは嗅覚を用いて情報の伝達と提示を行うデバイス
システムを開発し実験を行った. この実験は両方の鼻に香
りを送り濃度を変化させることで香りの強度, 空間情報を
伝達させることを示している.[6]
他にも味覚と嗅覚は複数の感覚と影響し合う研究は行わ
れおり, Nimesha らによる Vocktail は味覚, 嗅覚, 視覚を
利用した研究を行っている. 視覚情報として LED を使用
した色の印象, 嗅覚として香りを与え, 味覚として電気味覚
を使用しているこれらを使い水の風味がどのようにして変
化するのか実験を行った. このシステムは,3 つの感覚を活
用して味覚にちする影響を与えていたことを明らかにして
いる.[7]
