%%%%%%%%%%%%%%%%%%%%%%%%%%%%%%%%%%%%%%%%%%%%%%%%%%%%%%%%%%%%%%%%%%%%%%%%%%%%%%
%%% タイトル、著者、所属、概要

% 日本語タイトル
\jtitle{
レトロネーザルアロマを利用した味覚の拡張
}

% 英語タイトル
\etitle{
Taste expansion using retronasal aroma
}

% 日本語著者
% 共著者氏名の間隔は ~ や \quad 等で適宜調節のこと。
\jauthor{
濱家陸\({}^\dagger\) \quad
白須椋介\({}^\ddagger\) \quad
羽田久一\({}^\dagger\) \quad
}

% 英語著者
\eauthor{
Riku Hamaie\({}^\dagger\) \quad
Ryosuke Shirasu\({}^\ddagger\) \quad
Hisakazu Hada\({}^\dagger\) \quad
}

% 日本語所属
\jaffiliation{
\(\dagger\) 東京工科大学 メディア学部
	〒192-0982 東京都八王子市片倉町1404-1 
\(\ddagger\) 東京工科大学大学院バイオ・情報メディア研究科
  〒192-0982 東京都八王子市片倉町1404-1 
}

% 英語所属
\eaffiliation{
\(\dagger\) Tokyo University of Technology
\(\ddagger\) Graduate School of Bionics, Computer and Media Sciences,
	Tokyo University of Technology
}

% 連絡先電子メールアドレス
% (
% このサンプルでは「@」を2バイト文字にすることで対応してある。)
\email{
m01172238c@edu.teu.ac.jp
}

% 日本語概要
\jabstract{
レトロネーザルで感じた香りは味を感じ方に大きな影響を与えると考え,嗅覚から感じる風味に焦点をあてた実験を検証した。
口の中からの嗅覚刺激を用いて香りを提示することで, 風味を与えることができるのではないかと考え,香りを閉じこめた直径3センチメートルのゼリーを食べることで,風味を想起させることが出来るかを検討した.
被験者に風味を感じることができるかを判断してもらい,風味提示ゼリーの有用性を示した.	
}

% 英語概要
\eabstract{
In this study, we verified that the food's flavor changed when the aroma was presented from inside the mouth.
The results showed that the more the subjects were unable to concentrate, the more their faces shook.
As a result, it was shown that the aroma could change the taste from inside the mouth.
}

% 日本語キーワード
\jkeyword{
嗅覚, 味覚, 香り, 風味
}

% 英語キーワード
\ekeyword{
Smell, taste, fragrance, Flavor 
}
